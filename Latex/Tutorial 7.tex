\documentclass{article}
\usepackage[utf8]{inputenc}
\usepackage{amsmath}
\usepackage{setspace}
\usepackage{mathtools}
\usepackage{amssymb}
\usepackage{amsfonts}
\newcommand\der[2]{\frac{\partial{#1}}{\partial{#2}}}
\usepackage{sectsty}
\usepackage[parfill]{parskip}
\usepackage{changepage}   % for the adjustwidth environment
\usepackage{graphicx}
\graphicspath{ {./Pictures/} }
\usepackage{float}
\usepackage[margin=1in]{geometry}
\setlength{\parindent}{0em}
\sectionfont{\fontsize{12}{12}\selectfont}
\nonfrenchspacing
\renewcommand{\baselinestretch}{1.5}
\usepackage{indentfirst}
\usepackage{enumitem}
\setlist[itemize]{topsep=0pt,itemsep=0pt,partopsep=0pt,parsep=0pt}
\usepackage{xcolor}
\usepackage{titlesec}
\DeclareUnicodeCharacter{2212}{-}
\usepackage{tikz}
\usetikzlibrary{calc}
\newcommand{\tikzmark}[1]{\tikz[overlay,remember picture] \node (#1) {};}
\titleformat{\section}[block]{\color{blue}\Large\bfseries\filcenter}{}{1em}{}
\usepackage[normalem]{ulem}
\usepackage{calrsfs}
\renewcommand{\labelitemiv}{$\circledast$}
\renewcommand{\labelitemii}{$\circ$}
\titlespacing*{\subsection}{0pt}{0ex}{0ex}
\setlength{\parskip}{0.6em}

\title{ECON3510 Tutorial 7 Answers}
\date{2019}

\begin{document}

\maketitle

\section{Exercise 1}
\vspace{6mm}
\subsection{Question 1}

\begin{gather*}
  AC_{CH} (Q) = D_{CH}(Q) \\
  \frac{36}{Q+10} = 10 - Q \\
  36 = (10 - Q) (Q + 10) \\
  36 = 10Q + 100 - Q^{2} -10Q \\
  Q^{2} = 64 \\
  Q_{CH} = 8 \\
  P = 10 - 8 = 2
\end{gather*}

\par \vspace{0.8em}
\subsection{Question 2}

\begin{gather*}
  AC_{US} (Q) = D_{US}(Q) \\
  \frac{51}{Q + 10} = 10 - Q \\
  Q^{2} = 49 \\
  Q_{US} = 7 \\
  P = 10 - 7 = 3
\end{gather*}

\par \vspace{0.8em}
\subsection{Question 3}

World Demand is:
\begin{gather*}
  Q = (10 - P) + (10 - P) \\
  Q = 20 - 2P \\
  P = 10 - Q/2 \\
  D^{w}(Q) = 10 - Q/2
\end{gather*}
Set AC equal to inverse of world demand:
\begin{gather*}
  AC_{CH}(Q) = D^{W}(Q) \\
  \frac{36}{Q + 10} = 10 - Q/2 \\
  36 = (10 - 0.5Q)(Q + 10) \\
  -64 = 5Q - 0.5Q^{2} \\
  \therefore 0 = -Q^{2} + 10Q + 128
\end{gather*}
Using quadratic formula we have:
\begin{gather*}
  x = \frac{-b +/- \sqrt{b^{2} - 4ac}}{2a} \\
  Q = \frac{-10 +/- \sqrt{10^{2} - 4 \cdot -1 \cdot 128}}{2 \cdot -1} \\
  \therefore Q = 17.369 \\
  \therefore P = 10 - 17.369/2 = 1.32
\end{gather*}

\par \vspace{0.8em}
\subsection{Question 4}

\begin{gather*}
  AC_{US} = D^{W}(Q) \\
  \frac{51}{Q + 10} = 10 - Q/2 \\
  51 = (10 - 0.5Q)(Q + 10) \\
  \therefore 0 = -Q^{2} + 10Q + 98 \\
  \text{Using wolfram alpha:} \\
  Q = 16.09 \\
  P = 10 - 16.09/2 = 1.95
\end{gather*}

\section{Exercise 2}
\vspace{6mm}
\subsection{Question 1}

\begin{gather*}
  Q = (5-P/2) + (5-P/2) \\
  Q = 10 - P \\
  \therefore D^{W}(Q) = 10 - Q \\
  AC_{US} = D^{W}(Q) \\
  \frac{51}{Q+10} = 10 - Q \\
  \therefore Q = 7 \\
  \therefore P = 3 \\
  Q_{CH} = Q_{US} = 5 - 3/2 = 3.5
\end{gather*}

\par \vspace{0.8em}
\subsection{Question 2}

\begin{gather*}
  AC_{US} = \frac{51 - 2Q_{1}}{Q + 10} = \frac{51 - 14}{Q + 10} = \frac{37}{Q + 10} > AC_{CH} = \frac{36}{Q + 10} \\
  AC_{CH} = D^{W}(Q) \\
  \frac{36}{Q+10} = 10 - Q \\
  \therefore Q = 8 \\
  \therefore P = 2 \\
  Q_{CH} = Q_{US} = 5 - 2/2 = 1
\end{gather*}

\par \vspace{0.8em}
\subsection{Question 3}

\begin{gather*}
  AC_{US} = \frac{51 - 3Q_{1}}{Q + 10} = \frac{51 - 21}{Q + 10} = \frac{30}{Q + 10} < AC_{CH} = \frac{36}{Q + 10}
\end{gather*}
Therefore the US becomes the world producer:
\begin{gather*}
  AC_{US} = D^{W}(Q) \\
  \frac{30}{Q+10} = 10 - Q \\
  30 = (10 - Q) (10 + Q) \\
  \therefore Q = 8.37 \\
  \therefore P = 1.63 \\
  \therefore Q_{CH} = Q_{US} = 55 - 1.63/2 = 4.185
\end{gather*}



\end{document}
