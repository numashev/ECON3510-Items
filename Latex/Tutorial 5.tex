\documentclass{article}
\usepackage[utf8]{inputenc}
\usepackage{amsmath}
\usepackage{setspace}
\usepackage{mathtools}
\usepackage{amssymb}
\usepackage{amsfonts}
\newcommand\der[2]{\frac{\partial{#1}}{\partial{#2}}}
\usepackage{sectsty}
\usepackage[parfill]{parskip}
\usepackage{changepage}   % for the adjustwidth environment
\usepackage{graphicx}
\graphicspath{ {./Pictures/} }
\usepackage{float}
\usepackage[margin=1in]{geometry}
\setlength{\parindent}{0em}
\sectionfont{\fontsize{12}{12}\selectfont}
\nonfrenchspacing
\renewcommand{\baselinestretch}{1.5}
\usepackage{indentfirst}
\usepackage{enumitem}
\setlist[itemize]{topsep=0pt,itemsep=0pt,partopsep=0pt,parsep=0pt}
\usepackage{xcolor}
\usepackage{titlesec}
\DeclareUnicodeCharacter{2212}{-}
\usepackage{tikz}
\usetikzlibrary{calc}
\newcommand{\tikzmark}[1]{\tikz[overlay,remember picture] \node (#1) {};}
\titleformat{\section}[block]{\color{blue}\Large\bfseries\filcenter}{}{1em}{}
\usepackage[normalem]{ulem}
\usepackage{calrsfs}
\renewcommand{\labelitemiv}{$\circledast$}
\renewcommand{\labelitemii}{$\circ$}
\titlespacing*{\subsection}{0pt}{0ex}{0ex}
\setlength{\parskip}{0.6em}

\title{ECON3510 Tutorial 5 Answers}
\date{2019}

\begin{document}

\maketitle

\section{Exercise 1}
\vspace{6mm}
\subsection{Question 1}


$MPL_{F} = 200 - L_{f}, MPL_{C} = 100 - 2L_{C}, L = 200$ \\
From our equilibrium condition:
\begin{gather*}
  P_{C} MPL_{C} = w = P_{F}MPL_{F} \\
  2(100 - 2L_{C}) = 1(200 - L_{F}) \\
  200 - 4L_{C} = 200 - L_{F} \\
  4 L_{C} = L_{F} \ \ (*)
\end{gather*}
Since total labor is $L_{C} + L_{F} = 200$ we can substitute (*) in for $L_{F}$ to solve for $L_{C}$:
\begin{gather*}
  L_{C} + 4 L_{C} = 200 \\
  L_{C} = 40
\end{gather*}
By (*) since $L_{C} = 40$ it must be that:
\begin{gather*}
L_{F} = 4(40) = 160
\end{gather*}
Since we know $L_{C}$ and $L_{F}$ we know wages from using the wage formula:
\begin{gather*}
  w = 2(100 - 2(40)) \\
  w = 40
\end{gather*}


\par \vspace{0.8em}
\subsection{Question 2}

If prices double nothing change in terms of employment between sectors, however wages will double to 80. You can see this in the wage equation where:
\begin{gather*}
  W = 2 \times P_{1}MPL_{1}
\end{gather*}
However in the equilibrium equation we have:
\begin{gather*}
  2 \ times P_{1} MPL_{1} = 2 \times P_{2} MPL_{2}
\end{gather*}
So the multiplication by 2 cancels out and the equilibrium equation for employment does not change

\par \vspace{0.8em}
\subsection{Question 3}

The price for one good in our equilibrium condition changes so we have
\begin{gather*}
  2(100 - 2L_{C}) = 2(200 - L_{F}) \\
  200 - 4L_{C} = 400 - 2L_{F} \\
  2L_{F} = 200 + 4L_{C} \\
  L_{F} = 100 + 2L_{C} \ \ (**)
\end{gather*}
Since $L_{C} + L_{F} = 200$, using (**) we have:
\begin{gather*}
  L_{C} + L_{F} = 200 \\
  L_{C} + 100 + 2L_{C} = 200 \\
  3L_{C} = 100 \\
  L_{C} = 100/3
\end{gather*}
Using (**) again we have:
\begin{gather*}
  L_{F} = 100 + 2(100/3) = 500/3 = 166.666
\end{gather*}
Again using the wage formula:
\begin{gather*}
  W = P_{C}MPL_{C} \\
  W = 2(100 - 2L_{C}) \\
  W = 2(100 - 2(100/3)
\end{gather*}


\section{Exercise 2}
\vspace{6mm}
\subsection{Question 1}

Using that $MPL_{L} = \tfrac{\partial Q_{C}}{\partial L_{C}}$ we have:
\begin{gather*}
  Q_{C} = 4L_{C}^{0.5} \\
  MPL_{C} = 2L_{C}^{-0.5} \\
  MPL_{C} = \tfrac{2}{\sqrt{L_{C}}} = \tfrac{2}{L_{C}^{0.5}}
\end{gather*}
Similarly for food:
\begin{gather*}
  Q_{F} = 2L_{F}^{0.5} \\
  MPL_{F} = 1LF^{-0.5} \\
  MPL_{F} = \tfrac{1}{\sqrt{L_{F}}} = \tfrac{1}{L_{F}^{0.5}}
\end{gather*}

\par \vspace{0.8em}
\subsection{Question 2}

From our OC Formula:
\begin{gather*}
  OC_{C} = \tfrac{1}{LF^{0.5}} / \tfrac{2}{L_{C}^{0.5}} \\
  OC_{C} = \tfrac{1}{L_{F}^{0.5}} \times \tfrac{L_{C}^{0.5}}{2} \\
  OC_{C} = \tfrac{L_{C}^{0.5}}{2L_{F}^{0.5}}
\end{gather*}
Our production function for quantity equations were:
\begin{gather*}
  Q_{C} = 4L_{C}^{0.5} \Rightarrow L_{C}^{0.5} = \tfrac{Q_{C}}{4} \\
  Q_{F} = 2L_{F}^{0.5} \Rightarrow L_{F}^{0.5} = \tfrac{Q_{F}}{2}
\end{gather*}
Now we can substitute in L for Q in the OC equation to get:
\begin{gather*}
  OC_{C} = \tfrac{Q_{C}}{4} / 2(\tfrac{Q_{F}}{2})  \\
  OC_{C} = \tfrac{Q_{C}}{4Q_{F}}
\end{gather*}

\par \vspace{0.8em}
\subsection{Question 3}

Assuming $L_{C} = 125$ or $L_{F} = 125$ we get the following maximum quantities:
\begin{gather*}
  Q_{C} = 4(125)^{0.5} = 44.72 \\
  Q_{F} = 2(125)^{0.5} = 22.36
\end{gather*}

\par \vspace{0.8em}
\subsection{Question 4}

We use the following $PFQ_{C}, PFQ_{F}$, and $L$ equations to solve for the PPF:
\begin{gather*}
  (1) \ \ Q_{C} = 4L_{C}^{0.5} \Rightarrow L_{C} = \tfrac{Q_{C}^{2}}{16} \\
  (2) \ \ Q_{F} = 2L_{F}^{0.5} \Rightarrow L_{F} = \tfrac{Q_{F}^{2}}{4} \\
  (3) \ \ L_{F} + L_{C} = 125
\end{gather*}
Substituting equations (1) and (2) into (3) yields:
\begin{gather*}
  \tfrac{Q_{F}^{2}}{4} + \tfrac{Q_{C}^{2}}{16} = 125 \\
  Q_{F}^{2} + \tfrac{Q_{C}^{2}}{4} = 500 \\
  4Q_{F}^{2} + Q_{C}^{2} = 2000
\end{gather*}

\par \vspace{0.8em}
\subsection{Question 5}

We have that $OC_{C}^{*} = \tfrac{4Q_{C}^{*}}{Q_{F}^{*}}, PPF^{*} = Q_{F}^{*2} + 4Q_{C}^{*2} = 2000$. Our equilibrium conditions under free trade are:
\begin{gather*}
  (1) \ \ \tfrac{P_{C}^{W}}{P_{F}^{W}} = \tfrac{Q_{C}}{4Q_{F}} \\
  (2) \ \ \tfrac{P_{C}^{W}}{P_{F}^{W}} = (Q_{F} - X_{F})/(Q_{C} - X_{C}) \\
  (3) \ \ 4Q_{F}^{2} + Q_{C}^{2} = 2000 \\
  (4) \ \ \tfrac{P_{C}^{W}}{P_{F}^{W}} = 4 \tfrac{Q_{C}^{*}}{Q_{F}^{*}} \\
  (5) \ \ \tfrac{P_{C}^{W}}{P_{F}^{W}} = (Q_{F}^{*} + X_{F})/(Q^{*}_{C} + X_{C}) \\
  (6) \ \ Q_{F}^{*2} + 4Q_{C}^{*2} = 2000 \\
  (7) \ \ P_{C}X_{C} + P_{F}X_{F} = 0 \\
\end{gather*}

\underline{Quantities} \\
From conditions (1) and (4) we have:
\begin{gather*}
  \tfrac{Q_{C}}{4Q_{F}} = \tfrac{4Q_{C}^{*}}{Q_{F}^{*}} \\
  \text{Since } Q_{C}^{*} = Q_{F} \text{ (and vice versa) we have:} \\
  \tfrac{Q_{C}}{4Q_{F}} = \tfrac{4Q_{F}}{Q_{C}} \\
  \therefore Q_{C}^{2} = 16Q_{F}^{2} \ \ (*)  \\
  \therefore Q_{C} = 4Q_{F} \ \ (**)
\end{gather*}
Solving for quantities by using (*) in (3) we have:
\begin{gather*}
  4Q_{F}^{2} + Q_{C}^{2} = 2000 \\
  4Q_{F}^{2} + 16Q_{F}^{2} = 2000 \\
  20Q_{F}^{2} = 2000 \\
  Q_{F}^{2} = 100 \\
  \therefore Q_{F} = 10 \Rightarrow Q_{C}^{*} = 10
\end{gather*}
Using $Q_{F} = 10$ in condition (3) we have the following:
\begin{gather*}
  4(10)^{2} + Q_{C}^{2} = 2000 \\
  Q_{C}^{2} = 1600 \\
  \therefore Q_{C} = 40 \Rightarrow Q_{F}^{*} = 40
\end{gather*}

\underline{World Relative Price} \\
Substituting (**) into condition (1) gives us the world relative price:
\begin{gather*}
  \tfrac{P_{C}^{W}}{P_{F}^{W}} = \tfrac{4Q_{F}}{4Q_{F}} = 1 \\
  \text{Since } \tfrac{P_{C}^{W}}{P_{F}^{W}} = 1 \text{ we know from (7):} \\
  X_{F} = -X_{C} \ \ (***)
\end{gather*}

\underline{Exports} \\
Plugging equation (***) into condition (2) we can solve for exports:
\begin{gather*}
  \tfrac{P_{C}^{W}}{P_{F}^{W}} = (Q_{F} - X_{F})/(Q_{C} - X_{C}) \\
  1 = (Q_{F} - X_{F})/(4Q_{F} + X_{F}) \\
  4Q_{F} + X_{F} = Q_{F} - X_{F}  \\
  3Q_{F} = -2X_{F} \\
  -X_{F} = (\tfrac{3}{2})Q_{F}
\end{gather*}
Since $Q_{F} = 10$ we have:
\begin{gather*}
  -X_{F} = (\tfrac{3}{2})Q_{F} = (\tfrac{3}{2})10 = 15 \\
  \therefore -X_{F} = 15 \\
  \therefore X_{C} = 15
\end{gather*}

\underline{Summary}:$ Q_{F} = 10, Q_{C} = 40, Q_{F}^{*} = 40, Q_{C}^{*} = 10, \tfrac{P_{C}^{W}}{P_{F}^{W}} = 1, -X_{F} = 15, X_{C} = 15$


\end{document}
