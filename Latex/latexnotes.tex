\documentclass{article}
\usepackage[utf8]{inputenc}
\usepackage{amsmath}
\usepackage{setspace}
\usepackage{mathtools}
\usepackage{amssymb}
\usepackage{amsfonts}
\newcommand\der[2]{\frac{\partial{#1}}{\partial{#2}}}
\usepackage{sectsty}
\usepackage[parfill]{parskip}
\usepackage{changepage}   % for the adjustwidth environment
\usepackage{graphicx}
\graphicspath{ {./Pictures/} }
\usepackage{float}
\usepackage[margin=1in]{geometry}
\setlength{\parindent}{0em}
\sectionfont{\fontsize{12}{12}\selectfont}
\nonfrenchspacing
\renewcommand{\baselinestretch}{1.5}
\usepackage{indentfirst}
\usepackage{enumitem}
\setlist[itemize]{topsep=0pt,itemsep=0pt,partopsep=0pt,parsep=0pt}
\usepackage{xcolor}
\usepackage{titlesec}
\DeclareUnicodeCharacter{2212}{-}
\usepackage{tikz}
\usetikzlibrary{calc}
\newcommand{\tikzmark}[1]{\tikz[overlay,remember picture] \node (#1) {};}
\titleformat{\section}[block]{\color{blue}\Large\bfseries\filcenter}{}{1em}{}
\usepackage[normalem]{ulem}
\usepackage{calrsfs}
\renewcommand{\labelitemiv}{$\circledast$}
\renewcommand{\labelitemii}{$\circ$}
\titlespacing*{\subsubsection}{0pt}{0ex}{0ex}
\setlength{\parskip}{0.6em}

\title{ECON3510 Formula Sheet}
\date{2019}

\begin{document}

\maketitle

\section{Symbols}
$^{*}$ refers to foreign \\
$\text{variable}_{1},\text{variable}_{2}$ refers to goods $1$ and $2$ \\
$a_{i}$ refers to labor hours required to produce good i \\
$\text{variable}^{w}$ refers to world \\
$s$ refers to specialized good \\
$w$ refers to wage \\
$r$ refers to rental rate for capital \\
$PFQ$ refers to the production function of Q \\

\newpage

\section{Ricardian Equations in Terms of Home}
\underline{Gravity Model}: $T_{i,j} = \frac{A \times Y_{i} \times Y_{j}}{D_{i,j}}$ \\
\underline{Wage}: $w_{1} = \frac{P_{1}}{a_{1}}$ \\
\underline{Relative Wage}: $\frac{w}{w^{*}} = \frac{P_{1}}{a_{1}} \div \frac{P_{2}}{a_{2}^{*}} = \frac{P_{1}}{a_{1}} \cdot  \frac{a_{2}^{*}}{P_{2}}$ \\
\underline{Real Wages without Trade}: $w_{1}^{r} = \frac{w_{1}}{p_{1}} = \frac{1}{a_{1}}$
\begin{itemize}
  \item  \underline{Note}: this is equivalent to MPL
\end{itemize}
\underline{Real Wages with Trade}: $w_{1}^{r1} = \frac{w_{1}}{P_{1}^{w}} = \frac{1}{a_{1}}$ wages for good 1 in terms of specialized good 1, $w_{1}^{r2} = \frac{w_{1}}{P_{2}^{w}} = \frac{1}{a_{1}} \times \frac{P_{1}^{w}}{P_{2}^{w}}$ wages for good 1 in terms of non-specialized good 2
\begin{itemize}
  \item  \underline{Note}: make sure to use world prices
  \item  \underline{Derivation of Real Wages for Good 1 in Terms of Non-Specialized Good 2}: we have that $w_{1}^{r2} = \tfrac{w_{1}}{p_{2}^{w}}$, however we do not typically know the value of $p_{2}^{w}$. Instead, we usually only know the world price ratio $\tfrac{p_{1}^{w}}{p_{2}^{w}}$ is and so we need to find a solution for $w_{1}^{r2}$ in terms of $\tfrac{p_{1}^{w}}{p_{2}^{w}}$. Note that if we multiply both sides of our equation for $w_{1}^{r2}$ by $\tfrac{p_{1}^{w}}{p_{1}^{w}}$ we get:
  \begin{align*}
    w_{1}^{r2} &= \frac{w_{1}}{p_{2}^{w}} \cdot \frac{p_{1}^{w}}{p_{1}^{w}} \\
    w_{1}^{r2} &= \frac{w_{1}}{p_{1}^{w}} \cdot \frac{p_{1}^{w}}{p_{2}^{w}} \ \tag{*}
  \end{align*}
  Further note that from our equation for $w_{1}^{r1}$ we know that $\tfrac{w_{1}}{p_{1}^{w}} = \tfrac{1}{a_{1}}$. Inserting this into equation (*) yields the following soluton for real wages in terms of non-specialized good 2:
  \begin{gather*}
    w_{1}^{r2} = \frac{1}{a_{1}} \times \frac{p_{1}^{w}}{p_{2}^{w}}
  \end{gather*}
\end{itemize}
\underline{Relative Productivity}: $\frac{a_{1}}{a_{1}^{*}}$ \\
\underline{Marginal Rate of Substitution}: $MRS_{1,2} = \frac{MU_{1}}{MU_{2}} = \frac{P_{1}}{P_{2}}$ \\
\underline{Production}: $a_{1}Q_{1} + a_{2}Q_{2} = L$ \\
\underline{Production Possibility Frontier}: $Q_{1} = \frac{L}{a_{1}} - \frac{a_{2}}{a_{1}}Q_{2}$ \\
\underline{Marginal Productivity of Labor}: $MPL_{1} = \frac{1}{a_{1}}$ \\
\underline{Opportunity Cost}: $OC_{1} = \frac{a_{1}}{a_{2}}$ \\
\underline{Relative Price in Autarky}: $\frac{P_{1}}{P_{2}} = \frac{a_{1}}{a_{2}}$ \\
\underline{Relative Price in Free Trade}: $\frac{P_{1}}{P_{2}} = \frac{\text{total} \ Q_{1}}{\text{total} \ Q_{2}}$ where the quantity is the total produced in the economy \\
\underline{Autarky Equilibrium Occurs When}: $\frac{P_{1}}{P_{2}} = \frac{a_{1}}{a_{2}} = MRS_{1,2}$ \\
\underline{Closed Trade Specialization of Good 1 Occurs When}: $w_{1} = \frac{P_{1}}{a_{1}} > \frac{P_{2}}{a_{2}} = w_{2} \Rightarrow \frac{P_{1}}{P_{2}} > \frac{a_{1}}{a_{2}}$ \\
\underline{Free Trade Specialization of Good 1 (World Price is Not Given) Occurs When}:
 \begin{gather*}
   \frac{a_{1}}{a_{2}} < \frac{a_{1}^{*}}{a_{2}^{*}} \equiv wa_{1} < w^{*}a_{1}^{*} \equiv \frac{a_{1}^{*}}{a_{1}} > \frac{w}{w^{*}}
 \end{gather*}
\underline{Free Trade Specialization with World Price Given (three cases)}:
\begin{itemize}
  \item  \underline{Case 1}: $\frac{P_{1}}{P_{2}} = \frac{a_{1}}{a_{2}} < \frac{a_{1}^{*}}{a_{2}^{*}}$ then foreign specializes in good 2 and home does not specialize
  \item  \underline{Case 2}: $\frac{P_{1}}{P_{2}} < \frac{a_{1}}{a_{2}} < \frac{a_{1}^{*}}{a_{2}^{*}}$ then both home and foreign specialize in good 2
  \item  \underline{Case 3}: $\frac{a_{1}}{a_{2}} < \frac{P_{1}}{P_{2}} < \frac{a_{1}^{*}}{a_{2}^{*}}$ then home specializes in good 1 and foreign specializes in good 2
\end{itemize}

\newpage

\section{Heckscher-Ohlin Model}
\underline{Production Possibility Frontier}: $L = f(Q_{1}, Q_{2})$ \\
\underline{Isovalue Line - Representing Constant Value of Production (Indifference Curves)}: $V = P_{1}Q_{1} + P_{2}Q_{2}$
\begin{itemize}
  \item  \underline{Slope}: $-\tfrac{P_{1}}{P_{2}}$
\end{itemize}
\underline{Production}: occurs at the intersection of the most north-eastern isovalue line and the PPF, i.e. where $P_{C}/P_{F}$ equals the slop of the PPF  \\
\underline{MRS and Relative Price}: $P_{1}/P_{2} = \text{MRS}_{1,2}$ \\
\underline{Isoquant}: represents input possibilities in food production, where capital and labor inputs are imperfectly substitutible  \\
\underline{Relative Labor/Capital Demand}: if $\tfrac{L_{1}}{K_{1}} >  \tfrac{L_{2}}{K_{2}}$ then production of $1$ is relatively labor intensive and production of $2$ is relatively capital intensive
\begin{itemize}
  \item  \includegraphics[width=4cm, height=4cm]{pic1}
  \item  At any given wage/rental-ratio, production of 1 uses a higher labor-capital ratio since 1 is labor intensive and 2 is capital intensive
\end{itemize}
\underline{Wage Rental Ratio}: $\tfrac{w}{r}$
\begin{itemize}
  \item  \underline{Note}: changes in $\tfrac{w}{r}$ are tied to changes in $\tfrac{P_{1}}{P_{2}}$  \\
  \includegraphics[width=4cm, height=4cm]{pic2}
\end{itemize}
\underline{Stolper-Samuelson Theorem}: if the relative price of a good increases, then the real wage or rental rate of the factor used intensively in the production of that good increases, while the real wage or rental rate of the other factor decreases. Thus any change in the relative price of goods alters the distribution of income.
\begin{itemize}
  \item  \underline{Note}: if the relative price of cloth rises, the wage-rental ratio must rise. This will cause the labor-capital ratio used in the production of both goods to drop \\
  \includegraphics[width=6cm, height=4cm]{pic3}
\end{itemize}
\underline{Increase in Relative Price Effect} if the relative price for good 1 $P_{1}/P_{2}$ increase then this will:
\begin{itemize}
  \item  \underline{(1)}: raise income of workers relative to that of capital owners, $\tfrac{w}{r}$
  \item  \underline{(2)}: raise the ratio of capital to labor services, $\tfrac{K}{L}$ used in both industries
  \item  \underline{(3)}: raise the real income (purchasing power) of workers and lower the real income of capital owners
  \item  \underline{(4)}: raises the purchasing power of labor in terms of both goods while lowers the purchasing power of capital in terms of both goods
\end{itemize}
\underline{Rybczynski Theorem}: holding output prices constant, as the amount of a factor of production increases, the supply of the good that uses this factor intensively increases and the supply of the other good decreases
\begin{itemize}
  \item  \underline{Example}: an increase in the supply of labor shifts the economy’s production possibility frontier outward disproportionately in the direction of the more labour intense good's (cloth) production. At an unchanged  of cloth, the less labour intense good's (food) production declines \\
  \includegraphics[width=8cm, height=6cm]{pic4}
\end{itemize}
\underline{Trade Convergence}: trade leads to a convergence of s
\begin{itemize}
  \item  \underline{Example}: in the absence of trade, Home’s equilibrium would be at point 1, where domestic relative supply RS intersects the relative demand curve RD. Similarly, Foreign’s equilibrium would be at point 3. Trade leads to a world  that lies between the pretrade prices, such as at point 2 \\
  \includegraphics[width=8cm, height=6cm]{pic5}
\end{itemize}
\underline{Heckscher-Ohlin Theorem}: the country that is abundant in a factor exports the good whose production is intensive in that factors, so countries tend to export goods whose production is invensive in factors with which countries are abundantly endowed \\
\underline{Relative Consumption of Home}: $(Q_{1}-X_{1})/(Q_{2}-X_{2})$ \\
\underline{Relative Consumption of Foreign}: $(Q_{1}^{*}+X_{1})/(Q_{2}^{*}+X_{2})$ \\
\underline{Optimality Condition under Autarchy}: optimality under autarchy requires each countries  of good 1 to equal it's opportunity cost of production of good 1 and the marginal rate of substitution in consumption of good 1, so we have three conditions:
\begin{align*}
  \textit{Optimality in Production: }& \tfrac{P_{1}}{P_{2}} = f_{OC}(\tfrac{Q_{1}}{Q_{f}}) \\
  \textit{Optimality in Consumption: }& \tfrac{P_{1}}{P_{2}} = f_{MRS}(\tfrac{Q_{1}}{Q_{f}}) \\
  \textit{Production Possibility Frontier: }& L = f(Q_{1}, Q_{2})
\end{align*} \\
\underline{Optimality Condition under Free Trade}: optimality under free trade requires that the  in each country be the same and balance of payments to be zero, so we have six conditions:
\begin{align*}
  \textit{Optimality in Production Home: }& \tfrac{P_{1}^{w}}{P_{2}^{w}} = f_{OC}(\tfrac{Q_{1}}{Q_{f}}) \\
  \textit{Optimality in Consumption Home: }& \tfrac{P_{1}^{w}}{P_{2}^{w}} = f_{MRS}(\tfrac{(Q_{1}-X_{1})}{(Q_{f}-X_{2})} \\
  \textit{Production Possibility Frontier Home: }& L = f(Q_{1}, Q_{2}) \\
  \textit{Optimality in Production Foreign: }& \tfrac{P_{1}^{w}}{P_{2}^{w}} = f_{OC}(\tfrac{Q_{1}^{*}}{Q_{f}^{*}}) \\
  \textit{Optimality in Consumption Foreign: }& \tfrac{P_{1}^{w}}{P_{2}^{w}} = f_{MRS}(\tfrac{(Q_{1}^{*}+X_{1})}{(Q_{f}^{*}-X_{2})} \\
  \textit{Production Possibility Frontier Foreign: }& L = f(Q_{1}, Q_{2}) \\
  \textit{Balance of Payments: }& P_{1}^{w}X_{1} + P_{2}^{w}X_{2} = 0
\end{align*}

\newpage

\section{Specific Factors Model}
\underline{Main Features}: land (T) is used in the production of one good and capital (K) is used in the production of another, i.e. $Q_{C} = Q_{C}(K,L_{C})$ and $Q_{F}=Q_{F}(T,L_{F})$ \\
\underline{Total Labor}: $L_{1} + L_{2} = L$ \\
\underline{Production Possibility Frontier}: $f(PCQ_{1}) + f(PCQ_{2}) = L$
\begin{itemize}
  \item  \underline{Note}: in other words, we derive the production possibility frontier by using the following three questions, where we solve equations (1) and (2) for L and substitute them into (3)
  \begin{itemize}
    \item  \underline{(1)}: $PFQ_{1} = f(L_{1})$
    \item  \underline{(2)}: $PFQ_{2} = f(L_{2})$
    \item  \underline{(3)}: $L_{1} + L_{2} = L$
  \end{itemize}
\end{itemize}
\underline{Opportunity Cost}: equal to relative MPL, i.e. $OC_{1} = \tfrac{MPL_{2}}{MPL_{1}}$
\begin{itemize}
  \item  \underline{Quantity Terms}: typically we will want to substitute $Q$ in for $L$ to give us opportunity cost in terms of quantity produced, to do this we just solve our production function of $Q$ formula for $L$ and then substitute $L$ in!
\end{itemize}

\underline{Production Function and MPL}: $MPL_{1} = \tfrac{\partial PFQ_{1}}{\partial L_{1}}$, in other words, the marginal product of labor is the derivative of the production function with respect to labor \\
\underline{Labor Demand Curve}: $MPL_{1} \times P_{1} = W$, where wages are $W$ \\
\underline{Equilibrium in Autarky}: (1) wages are equal between sectors $MPL_{2} \times P_{2} = W = MPL_{1} \times P_{1}$, (2) PPF is tangent to  line $-\tfrac{MPL_{2}}{MPL_{1}} = -\tfrac{P_{1}}{P_{2}}$ \\
\underline{Trade Spending Constraint}: $P_{1}D_{1} + P_{2}D_{2} = P_{1}Q_{1} + P_{2}Q_{2}$, i.e. a country cannot spend more than it earns \\
\underline{Import from Trade} $\underbrace{D_{1} - Q_{1}}_{import \ of \ 1} = (\tfrac{P_{2}}{P_{1}}) \times (Q_{2} - D_{2})$ \\

\underline{Optimality Condition under Free Trade}: optimality under free trade requires that the  in each country be the same and balance of payments to be zero, so we have six conditions:
\begin{align*}
  \textit{Optimality in Production Home: }& \tfrac{P_{1}^{w}}{P_{2}^{w}} = f_{OC}(\tfrac{Q_{1}}{Q_{f}}) \\
  \textit{Optimality in Consumption Home: }& \tfrac{P_{1}^{w}}{P_{2}^{w}} = f_{MRS}(\tfrac{(Q_{1}-X_{1})}{(Q_{f}-X_{2})} \\
  \textit{Production Possibility Frontier Home: }& L = f(Q_{1}, Q_{2}) \\
  \textit{Optimality in Production Foreign: }& \tfrac{P_{1}^{w}}{P_{2}^{w}} = f_{OC}(\tfrac{Q_{1}^{*}}{Q_{f}^{*}}) \\
  \textit{Optimality in Consumption Foreign: }& \tfrac{P_{1}^{w}}{P_{2}^{w}} = f_{MRS}(\tfrac{(Q_{1}^{*}+X_{1})}{(Q_{f}^{*}-X_{2})} \\
  \textit{Production Possibility Frontier Foreign: }& L = f(Q_{1}, Q_{2}) \\
  \textit{Balance of Payments: }& P_{1}^{w}X_{1} + P_{2}^{w}X_{2} = 0
\end{align*}

\newpage

\section{Standard Trade Model}
\underline{Optimal Production}: a nation chooses $Q_{1}$ and $Q_{2}$ such that the value of its output $V = P_{1}Q_{1} + P_{2}Q_{2}$ is maximized
\begin{itemize}
  \item  \underline{Note}: this is where the PPF is tangent to the isovalue line $-(\tfrac{P_{C}}{P_{F}})$
\end{itemize}
\underline{Effect of Relative Price Increase}: an increase in $\tfrac{P_{1}}{P_{2}}$ makes; (1) the isovalue line steeper, (2) supply of good 1 relative to good 2 rises, (3) shift from $Q$ to $Q^{*}$
\begin{itemize}
  \item  \includegraphics[width=6cm, height=4cm]{pic6}
\end{itemize}
\underline{Consumption Optimality}: $P_{1}D_{1} + P_{2}D_{2} = P_{1}Q_{1} + P_{2}Q_{2} = V$, i.e. the value of consumption must equal the value of production, where $D$ is the demand/consumption and $Q$ is supply  \\
\underline{Relative Market Clearing}: $RS = RD$, i.e. relative supply must equal relative demand in a given country, and allows us to solve for price
\begin{itemize}
  \item  \underline{Note}: relative supply is equal to supply of good 1 divided by supply of good 2, the same applies for demand
\end{itemize}
\underline{MRS Optimality/Formula}: $MRS_{1,2} = \tfrac{P_{1}}{P_{2}}$ \\
\underline{World Real Supply of Good 1}: $World \ RS = Q_{1} + Q_{1}^{*}$
\begin{itemize}
  \item  \underline{Note}: $Q_{1}$ and $Q_{1}^{*}$ may be represented as a function to solve
\end{itemize}
\underline{World Relative Supply}: $World \ Relative \ RS =  \tfrac{Q_{1} + Q_{1}^{*}}{Q_{2} + Q_{2}^{*}}$
\begin{itemize}
  \item  \underline{Note}: this can be calculated by taking the world real supply of good 1 as a ratio of good 2 from the former equation where $RS^{w} = RS_{1}^{w} + RS_{2}^{w}$
\end{itemize}
\underline{World Relative Demand}: $World \ Relative \ RD = \tfrac{D_{1} +D_{1}^{*}}{D_{2}+D_{2}^{*}}$
\begin{itemize}
  \item  \underline{Note}: this is also equal to relative demand for good 1 and good 2 if it is the same in both countries, i.e. $\tfrac{D_{1}}{D_{2}} + \tfrac{D_{1}^{*}}{D_{2}^{*}} = \tfrac{D_{1} +D_{1}^{*}}{D_{2}+D_{2}^{*}}$ if $D_{2} = D_{2}^{*}, D_{1} = D_{1}^{*}$
\end{itemize}
\underline{Total World Quantity}: $\text{Total Q } = Q_{1} + Q_{1}^{*} + Q_{2} + Q_{2}^{*} = D_{1} + D_{1}^{*} + D_{2} + D_{2}^{*}$ \\
\underline{World Equilibrium for RS and RD}: occurs where $RS^{W} = RD^{W}$ where $RS$ and $RD$ are a function of relative price and relative quantity \\
\underline{Biased Growth}: growth is biased when it shifts production possibilities out more toward one good than toward another, which causes an asssociated shift in RS towards the biased good
\begin{itemize}
  \item  \underline{Note}: make sure to think about the math when trying to figure out which good had biased growth, use the PPF to check!!!
\end{itemize}
\underline{Real Interest Rate}: $\tfrac{1}{1+r}$ is determined by the intersection of world relative demand and world relative supply
\begin{itemize}
  \item  \includegraphics[width=4cm, height=4cm]{pic7}
\end{itemize}
\underline{Price of Current Consumption}: normalised to 1 so that $P_{c} = 1$ \\
\underline{Price of Future Consumption Relative to Current Consumption}: $P_{F} = \tfrac{1}{1+r}$, this is based on interest ($r$) and current price $P_{c}$
\begin{itemize}
  \item  \underline{Note}: this means that 1 unit of current consumption is worth $1 + r$ units of future consumption, or equivalently to get 1 unit of future consumption you must pay $P_{F} = \tfrac{1}{1+r}$ units of current consumption
\end{itemize}
\underline{Relative Price and Real Interest}: $\tfrac{P_{1}}{P_{2}} = 1 + r$
\begin{itemize}
  \item  \underline{Note}: this means that the isovalue lines are $-\tfrac{P_{1}}{P_{2}} = -(1+r)$
\end{itemize}
\underline{Intertemporal Budget Constraint}: $\frac{P_{1}}{P_{2}}D_{1} + D_{2} = (1+r)Q_{1} + Q_{2}$
\begin{itemize}
  \item  \underline{Note}: this can alternatively be written as $D_{2} = Q_{2} + (1+r)(Q_{1} - D_{1})$
\end{itemize}
\underline{Home Intertemporal Exports/Imports}: home exports $Q_{1} - D_{1}$ units of current consumption and imports are $D_{2} - Q_{2} = (1+r)(Q_{1} - D_{1})$ units of future consumption \\
\underline{Foreign Intertemporal Exports/Imports}: foreign exports $Q_{2}^{*}+D_{2}^{*} = (1+r)(D_{1}^{*} + Q_{1}^{*})$ units of future consumption and imports $D_{1}^{*}  - Q_{1}^{*}$ units of current consumptions

\newpage

\section{Economies of Scale}
\underline{One Period Autarky Equilibrium}: each country produces such that $AC = D(Q)$, i.e. average cost equals inverse demand \\
\begin{itemize}
  \item  \underline{Note}: $D(Q)$ is jusst the notation used for demand as a function of Q and can be replaced by $P$ in the solution
\end{itemize}
\underline{Free Trade Equilibrium}: the country with the lowest $AC$ will be the producer. That country produces such that $AC = D^{w}(Q)$, where $D^{w}(Q)$ is the sum of all countries demands
\begin{itemize}
  \item  \underline{Note}: without additional information we generally assume that if a country has a head start/initial advantage then they will be the producer
  \item  \underline{Consumption}: each country consumes according to their demand function
\end{itemize}

\newpage

\section{International Demand/Supply}
\underline{Equilibrium Condition for Price/Quantity}: set $Q_{S} = Q_{D}$, with both being a function of price \\
\underline{Consumer Surplus}: $CS = (C_{NG} \cdot Q_{D})/2$ where $C_{NG}$ is the difference between $P$ at $Q_{D} = 0$ and actual $P$ \\
\underline{Producer Surplus}: $P \cdot S_{NS} +  [(Q_{S} - S_{NS}) \cdot P]/2$ where $S_{NS}$ is the amount supplied at $P = 0$

\end{document}
