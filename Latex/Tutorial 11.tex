\documentclass{article}
\usepackage[utf8]{inputenc}
\usepackage{amsmath}
\usepackage{setspace}
\usepackage{mathtools}
\usepackage{amssymb}
\usepackage{amsfonts}
\newcommand\der[2]{\frac{\partial{#1}}{\partial{#2}}}
\usepackage{sectsty}
\usepackage[parfill]{parskip}
\usepackage{changepage}   % for the adjustwidth environment
\usepackage{graphicx}
\graphicspath{ {./Pictures/} }
\usepackage{float}
\usepackage[margin=1in]{geometry}
\setlength{\parindent}{0em}
\sectionfont{\fontsize{12}{12}\selectfont}
\nonfrenchspacing
\renewcommand{\baselinestretch}{1.5}
\usepackage{indentfirst}
\usepackage{enumitem}
\setlist[itemize]{topsep=0pt,itemsep=0pt,partopsep=0pt,parsep=0pt}
\usepackage{xcolor}
\usepackage{titlesec}
\DeclareUnicodeCharacter{2212}{-}
\usepackage{tikz}
\usetikzlibrary{calc}
\newcommand{\tikzmark}[1]{\tikz[overlay,remember picture] \node (#1) {};}
\titleformat{\section}[block]{\color{blue}\Large\bfseries\filcenter}{}{1em}{}
\usepackage[normalem]{ulem}
\usepackage{calrsfs}
\renewcommand{\labelitemiv}{$\circledast$}
\renewcommand{\labelitemii}{$\circ$}
\titlespacing*{\subsection}{0pt}{0ex}{0ex}
\setlength{\parskip}{0.6em}

\title{ECON3510 Tutorial 11 Answers - See Claudio's Answer Guide for Text/Missing Sections}
\date{2019}

\begin{document}

\maketitle

\section{Exercise 1}
\vspace{6mm}
\subsection{Question 1}

A's profit is:
\begin{gather*}
  \pi = (P - MC) \cdot Q_{A} = (130 - Q_{A} - Q_{B} - 10) \cdot Q_{A}
\end{gather*}
Taking FOC with respect to $Q_{A}$ we have:
\begin{gather*}
  0 = \frac{\partial \pi}{\partial Q_{A}} = (130 - Q_{A} - Q_{B}  - 10) + (-Q_{A}) \ \ \  \ \  \ (1)
\end{gather*}
Since the maximization problem is symmetrical for B so we have:
\begin{gather*}
  0 = 120 - 2Q_{B} - Q_{A} \ \ \  \ \  \  (2)
\end{gather*}
Setting equation (1) = (2) we get:
\begin{gather*}
  120 - 2Q_{A} - Q_{B} = 120 - 2Q_{B} - Q_{A} \\
  \therefore Q_{A} = Q_{B} \ \ \ \ \ \ \ (*)
\end{gather*}
We can plug equation (*) into (1) to find:
\begin{gather*}
  0 = 120 - 2Q_{A} - Q_{B} \\
  0 = 120 - 3Q_{A} \\
  Q_{A} = 40 \\
  Q_{B} = 40 \\
  P = 130 - Q_{A} - Q_{B} = 130 - 80 = 50 \\
  \pi_{A} = \pi_{B} = (P - MC)(Q_{A}) = (50 - 10) \cdot 40 = 1600
\end{gather*}

\vspace{6mm}
\subsection{Question 2}

A's profit is now:
\begin{gather*}
  \pi = (130 - Q_{A} - Q_{B} - 4) Q_{A}
\end{gather*}
Take FOC with respect to $Q_{A}$ we have:
\begin{gather*}
  0 = 126 - 2Q_{A} - Q_{B} \ \ \ \ \ \ (1)
\end{gather*}
B's profit has not changed so:
\begin{gather*}
  0 = 120 - 2Q_{B} - Q_{A} \ \ \ \  \ \ \ (2)
\end{gather*}
Setting (1) = (2) to yield:
\begin{gather*}
  126 - 2Q_{A} - Q_{B} = 120 - 2 Q_{B}  - Q_{A} \\
  6 - Q_{A} = - Q_{B} \\
  Q_{B} = Q_{A} - 6 \ \ \ \ \ \ \ (*)
\end{gather*}
Substitution equation (*) into (1) to get the following:
\begin{gather*}
  0 = 126 - 2Q_{A} - Q_{A} + 6 \\
  3Q_{A} = 132
  \therefore Q_{A} = 44 \\
  Q_{B} = 38 \\
  P = 130 - 38 - 44 = 48 \\
  \pi_{A} = (P - MC) Q_{A} = (48-4)44 = 1936 \\
  \pi_{B} = (P - MC)Q_{B} = (48-10)44 = 1444 \\
  \Delta Revenue EU = 0 - 44 \cdot 6 = -264 \\
  \Delta WelfareEU = \Delta \pi_{A} + \Delta Rev = (1936 - 1600) - 264 = 72 \\
  \Delta WelfareUS = \Delta \pi_{B} = 1444 - 1600 = -156 \\
\end{gather*}

\vspace{6mm}
\subsection{Question 3}

Since B also applies the subsidy it has a symmetrical FOC with firm A so that:
\begin{gather*}
  0 = 126 - 2Q_{A} - Q_{B} \ \ \ \ \ \ \ (1) \\
  0 = 126 - 2Q_{B} - Q_{A} \ \ \ \ \ \ \ (2)
\end{gather*}
Setting (1) = (2) we get the following:
\begin{gather*}
  126 - 2Q_{A} - Q_{B} = 126 - 2Q_{B} - Q_{A} \\
  Q_{A} = Q_{B} \ \ \ \ \ \ (*)
\end{gather*}
Substituting (*) into (1) yields:
\begin{gather*}
  0 = 126 - 2Q_{A} - Q_{A} \\
  3Q_{A} = 126 \\
  Q_{A} = 42 \\
  Q_{B} = 42 \\
  P = 130 - 42 - 42 = 46 \\
  \pi_{A} = \pi_{B} = (46-4)42 = 1764  \\
  RevenueLossEU = RevenueLossUS = 42 \times 6 = 252 \\
  \Delta Welfare US (rel1) = 1764 - 1600 - 252 = - 88 \\
  \Delta Welfare EU (rel1) = 1764 - 1600 - 252 = - 88 \\
  \Delta Welfare US (rel2) = 1764 - 1444 - 252 = 68 \\
  \Delta Welfare EU (rel2) = 1764 - 1936 + 12 = -160
\end{gather*}
Note that both countries will be better off in the no subsidy case (question 1), however if we have case 2 with one subsidy then the US will retaliate with case 3


\section{Exercise 2}
\vspace{6mm}

See Claudio's answer guide or the tutorial recording




\end{document}
